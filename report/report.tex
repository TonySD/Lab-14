\documentclass[a4paper, 12pt]{article}
\usepackage{geometry}
\usepackage[russian]{babel}
\usepackage[T2A]{fontenc}
\usepackage{chngcntr}
\usepackage{graphicx}
\usepackage{verbatim}

\begin{document}


\begin{titlepage}

\begin{center}
{\textsc{\textbf{Правительство Российской Федерации}}}\\
\vspace{0.5cm}
\hrule
\vspace{0.5cm}
{\textsc{Федеральное государственное автономное образовательное учреждение\\высшего образования <<Национальный исследовательский университет\\<<Высшая школа экономики>>}}\\
\vspace{1cm}
Кафедра <<Компьютерная безопасность>>
\end{center}

\vspace{\fill}
% Поменять номер лабы
\begin{center}
{\Large{\textbf{ОТЧЕТ \\ К ЛАБОРАТОРНОЙ РАБОТЕ №14}}} \\
\vspace{1em}
{\textbf{по дисциплине}} \\
\vspace{1em}
{\large{\textbf{<<Языки программирования>>}}}
\end{center}

\vspace{\fill}


\begin{flushright}
  \begin{minipage}[center]{15cm}

    \begin{minipage}[left]{5cm}
      {Работу выполнил\\студент группы СКБ-222}
    \end{minipage}
    \begin{minipage}[center]{5cm}
      \vspace{1.25cm}
      \hrulefill\\[-1cm]
      \begin{center}{подпись, дата}\end{center}
    \end{minipage}
    \begin{minipage}[right]{4cm}
      \vspace{0.4cm}
      \begin{flushright}{А.С. Вагин}\end{flushright}
    \end{minipage}
    \\
    \\
    \\
    \begin{minipage}[left]{5cm}
      {Работу проверил}
    \end{minipage}
    \begin{minipage}[center]{5cm}
      \vspace{1.25cm}
      \hrulefill\\[-1cm]
      \begin{center}{подпись, дата}\end{center}
    \end{minipage}
    \begin{minipage}[right]{4cm}
      \begin{flushright}{С.А. Булгаков}\end{flushright}
    \end{minipage}
  \end{minipage}
\end{flushright}

\vspace{\fill}

\begin{center}
Москва~2023
\end{center}
\end{titlepage}
\setcounter{page}{2}
\setcounter{secnumdepth}{5}
\setcounter{tocdepth}{5}

% Содержание
\tableofcontents
\cleardoublepage

\setcounter{section}{1}
\counterwithout{subsection}{section}
\graphicspath{ {./images/} }

% Постановка задачи
\cleardoublepage
\section*{Постановка задачи}\addcontentsline{toc}{section}{Постановка задачи}

\begin{verbatim}
Разработать консольную утилиту позволяющую выполнять проверку целостности 
файлов на основе механизма контрольных сумм. При запуске программа проверяет 
наличие файла `Checksum.ini`. При его наличия выполняется проверка 
контрольных сумм для файлов указанных в нем, иначе если поток ввода не пуст, 
то из него считывается имя файла(ов) и, отделенные символом табуляции, 
их контрольные суммы (см. ключ `-a`). Если поток ввода пуст, проводится разбор 
параметров командной строки и, в зависимости от входных данных, выполняется 
действие либо выводится `usage`. 

Режимы запуска программы:
    * без параметров - использование файла `Checksum.ini`;
    * `-a algorithm`, где `algorithm` одно из `crc32`, `md5`, `sha256` - задаёт 
    алгоритм для рассчёта контрольной суммы;
    * `файл...` - имена файлов для вычисления контрольных сумм (см. ключ `-a`).

Пример содержимого файла `Checksum.ini`:
``` 
; This is comment 
[CRC32] 
README.md=0x720C65BA 
``` 
\end{verbatim}

\cleardoublepage



\section*{Основная часть}\addcontentsline{toc}{section}{Основная часть}

\subsection{Описание хэш-функций}
Для данной лабораторной работы были взяты готовые реализации требующихся
хэш-функций, а именно SHA256, CRC32 и MD5. \\
Во всех классах была реализована функция \textit{hash}, которая принимала на вход
массив типа \textit{unsigned char} и его длина, и возвращала \textit{std::string},
который является хэшем в шестнадцатеричном формате. \\
Также был создан класс \textit{HashFactory} для упрощенного доступа ко всем функциям. \\
Класс \textit{HashFactory} принимает на вход нужные для хэширования данные, 
а также аргумент из класса перечислений \textit{Hashes}, для определения
нужной хэш-функции.

\subsection{Описание функций}

\subsubsection{Вспомогательные функции для класса перечислений \textit{Hashes}}
Были разработаны функции \textit{algorithmToText} и \textit{parseAlgorithm} для
преобразования аргумента класса в текст и обратно соответственно.

\subsubsection{Функция для получения файла в байтовом представлении}
Функция \textit{getBytesFromFile} получает на вход имя файла в виде \textit{std::string}.
Если файла не существует, вызывается ошибка. Функция возвращает \textit{std::vector} с
байтами файла в формате \textit{unsigned char}. 
% Про новые функции

\cleardoublepage


\section*{Приложение A}\addcontentsline{toc}{section}{Приложение А}
\renewcommand\thesection{\Alph{section}}
\renewcommand\thesubsection{\thesection.\arabic{subsection}}
\setcounter{subsection}{0}

\subsection{UML-диаграмма \textit{clearStdin}}
% \includegraphics[width=\columnwidth]{clearStdin.png}


\cleardoublepage

\setcounter{subsection}{0}
\section*{Приложение B}\addcontentsline{toc}{section}{Приложение B}
\renewcommand\thesection{\Alph{section}}
\renewcommand\thesubsection{B.\arabic{subsection}}

\subsection{Код программы}

\fontsize{9}{9}\selectfont
% Код сюда
\begin{verbatim}

\end{verbatim}

\end{document}